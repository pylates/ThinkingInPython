\section*{\texorpdfstring{Y: Iteradores: \newline  Algoritmos de desacoplamiento de contenedores}{Y: Iteradores: Algoritmos de desacoplamiento de contenedores}}
\label{sec:yiadddc}
\addcontentsline{toc}{section}{\nameref{sec:yiadddc}}


\textit{Alexander Stepanov pensó durante años sobre el problema de las técnicas de programación genéricas antes de crear el STL (Standard Template Library) (junto con Dave Musser). Llegó a la conclusión de que todos los algoritmos están definidos en las estructuras algebraicas – lo que llamaríamos contenedores.}

En el proceso, Alexander Stepanov se dio cuenta que los iteradores son fundamentales para el uso de algoritmos, porque desacoplan los algoritmos del tipo especifico de contenedor con que el algoritmo actualmente podría estar trabajando. Esto significa que usted puede describir el algoritmo sin preocuparse de la secuencia particular en la cual está operando. En general, \textit{cualquier} código que usted escribe utilizando iteradores es desacoplado de la estructura de datos que el código está manipulando, y por lo tanto su código es más general y reutilizable. \newline

El uso de iteradores también amplía su código en el campo de \textit{programación funcional}, cuyo objetivo es describir lo que un programa está haciendo a cada paso en lugar de cómo lo está haciendo. Es decir, usted dice "el tipo" en lugar de describir el tipo. El objetivo del STL de C ++ era proporcionar este enfoque a la programación genérica de C ++ (cómo el éxito de este enfoque será en la realidad, aún está por verse). \newline

Si ha utilizado contenedores en Java (y es difícil escribir código sin usarlos), ha utilizado iteradores – en forma del \textbf{Enumeration} en Java 1.0/1.1 y de \textbf{Iterator} en Java 2.0. Así que usted ya debe estar familiarizado con su uso general.\footnote{Si no, consulte el Capítulo 9, \textit{Holding Your Objects  (Manteniendo Sus objetos)}, en \textit{Iterators in Thinking in Java} segunda edición (descargable gratuitamente desde \url{www.bruceeckel.com/}).} \newline

Debido a que en Java 2.0 los contenedores dependen en gran medida de los iteradores, se convierten en excelentes candidatos para las técnicas de programación genéricas / funcionales. Este capítulo explorará estas técnicas mediante la conversión de los algoritmos de STL para Java, para su uso con la librería de contenedor Java 2.  \newline


\subsection*{Iteradores Type-safe}
\label{subsec:icsdt}
\addcontentsline{toc}{subsection}{\nameref{subsec:icsdt}}


En \textit{Thinking in Java, segunda edición}, muestro la creación de un contenedor Type-safe que sólo aceptará un tipo particular de objeto. Un lector, Linda Pazzaglia, preguntó por el otro componente obvio type-safe, un iterador que trabajaría con los contenedores básicos \textbf{java.util}, pero imponer la restricción de que el tipo de objetos sobre los que itera sea de un tipo particular. \newline

Si Java siempre incluye un mecanismo de plantilla, este tipo de iterador tendrá la ventaja añadida de ser capaz de devolver un tipo específico de objeto, pero sin las plantillas se ve obligado a retornar \textbf{Objects} genéricos, o requerir un poco de codificación manual para cada tipo que desea iterar. Tomaré el enfoque anterior.     \newline

Una segunda decisión de diseño involucra el tiempo en que se determina el tipo de objeto. Un enfoque consiste en tomar el tipo del primer objeto que el iterador encuentra, pero esto es problemático debido a que los contenedores pueden arreglar de nuevo los objetos de acuerdo con un mecanismo de ordenamiento interno (tal como una tabla hash) y por lo tanto es posible obtener diferentes resultados de una iteración a la siguiente. El enfoque seguro es exigir al usuario establecer el tipo durante la construcción del iterador.     \newline

Por último, ¿cómo construir el iterador? No podemos reescribir las librerías de clases Java existentes que ya producen \textbf{Enumerations} e \textbf{Iterators}.  Sin embargo, podemos utilizar el patrón de diseño \textit{Decorator}, y crear una clase que simplemente envuelve el \textbf{Enumeration} o \textbf{Iterator} que se produce, generando un nuevo objeto que tiene el comportamiento de iteración que queremos (que es, en este caso, lanzar un \textbf{RuntimeException} si se encuentra un tipo incorrecto) pero con la misma interfaz que el \textbf{Enumeration} original o \textbf{Iterator}, de modo que se puede utilizar en los mismos lugares (puede argumentar que esto es en realidad un patrón \textit{Proxy}, pero es más probable \textit{Decorator} debido a su intención). Aquí está el código: \newline

\begin{lstlisting} 
# util:TypedIterator.py 

class TypedIterator(Iterator): 
  private Iterator imp 
  private Class type 
  def __init__(self, Iterator it, Class type): 
    imp = it 
    self.type = type 
    
  def hasNext(self):  
    return imp.hasNext()  
    
  def remove(self): imp.remove()  
  def next(self): 
    Object obj = imp.next() 
    if(!type.isInstance(obj)) 
      throw ClassCastException( 
        "TypedIterator for type " + type + 
        " encountered type: " + obj.getClass()) 
    return obj 
# :~ 
\end{lstlisting}
