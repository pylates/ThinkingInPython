\newpage
\thispagestyle{empty}


La composición de esta traducción se realizó utilizando \LaTeX, \\(gracias al editor online ShareLatex)\footnote{\url{www.sharelatex.com}}. \\

El lector es totalmente libre de hacer correcciones (en caso de algún error de sintaxis en la traducción de inglés a español) dentro de la traducción en beneficio de la comunidad. \\

\begin{center}

El código fuente de esta traducción se encuentra en: \\
\url{https://github.com/LeidyAldana/ThinkingInPython}  \\
y su respectivo ejecutable en: \\ \url{https://es.slideshare.net/glud/traduccin-thinking-in-python-62703684}\\
\vspace{0.6cm}
Además, si el lector requiere de los ejemplos planteados, estos se encuentran disponibles en:\\ \url{https://github.com/LeidyAldana/ThinkingInPython/tree/master/Ejemplos}\\

\end{center}

\vspace{0.6cm}

Adicionalmente, el autor del libro tiene la siguiente opinión frente a la traducción:

\vspace{0.6cm}

\includegraphics[width=\textwidth]{Words}

\vspace{0.6cm}

\textit{ “Supongo que esto es útil para usted. Ese libro nunca fue terminado y lo que yo escribiría ahora es muy diferente. Pero si funciona para usted, eso está bien."  }   

\begin{flushright}

09 de Enero de 2016. Enviado por Bruce Eckel, vía Gmail.
    
\end{flushright}

% flushright == alineado a la izquierda

\newpage

\begin{center}

\textbf{Grupo GNU Linux Universidad Distrital. \\
Semillero de Investigación en Tecnología Libre.} \\
\url{https://glud.org/}

\vspace{0.6cm}

\textbf{Traducido por : \\
Leidy Marcela Aldana Burgos. \\}
LeidyMarcelaAldana@gmail.com \\
\textbf{(Estudiante de Ingeniería de Sistemas)} \\

\vspace{0.6cm}
\textbf{Universidad Distrital Francisco José de Caldas.} \\
\url{www.udistrital.edu.co}  \\
\vspace{0.6cm}
\textbf{Correcciones hechas por:\\ José Noé Poveda}\\pylatex@gmail.com\\\textbf{Docente Facultad de Ingeniería.\\Coordinador Grupo GNU Linux Universidad Distrital }

\vspace{0.6cm}

\textbf{Bogotá, Colombia.\\2017}

\vspace{0.9cm}

\includegraphics[width=\textwidth]{GLUDing.jpg}

\end{center}

\newpage