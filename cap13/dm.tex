\section*{11 : Despacho Múltiple}
\label{sec:dm}
\addcontentsline{toc}{section}{\nameref{sec:dm}}

Cuando se trata de múltiples tipos que están interactuando, un programa puede un programa puede obtener todo desordenado. Por ejemplo, considere un sistema que analiza y ejecuta expresiones matemáticas. Usted desea ser capaz de decir \textbf{Number + Number}, \textbf{Number * Number}, etc., donde \textbf{Number} es la clase base para una familia de objetos numéricos. Pero cuando usted dice \textbf{a + b}, y no conoce el tipo exacto de uno u otro \textbf{a} o \textbf{b}, así que ¿cómo se puede conseguir que interactúen correctamente?  \newline

La respuesta comienza con algo que probablemente no piensas: Python sólo realiza despacho individual. Es decir, si está realizando una operación en más de un objeto cuyo tipo es desconocido, Python puede invocar el mecanismo de enlace dinámico a uno sólo de esos tipos. Esto no resuelve el problema, por lo que termina detectando algunos tipos manualmente y produciendo con eficacia su propio comportamiento de enlace dinámico.   \newline

La solución es llamada \textit{multiple dispatching : despacho múltiple}. Recuerde que el polimorfismo puede ocurrir sólo a través de llamadas a funciones miembro, así que si quiere que el doble de despacho ocurra, debe haber dos llamadas a la función miembro: el primero para determinar el primero de tipo desconocido, y el segundo para determinar el segundo de tipo desconocido. Con despacho múltiple, usted debe tener una llamada a un método polimórfico para determinar cada uno de los tipos. Generalmente, va a configurar una configuración tal que una sola llamada de función miembro produce más de una dinámica llamada a la función miembro y por lo tanto determina más de un tipo en el proceso. Para obtener este efecto, usted necesita trabajar con más de una llamada a un método polimórfico: usted necesitará una llamada para cada despacho. Los métodos en el siguiente ejemplo se llaman \textbf{compete( )} y \textbf{eval( )}, y son ambos miembros del mismo tipo. (En este caso habrá sólo dos despachos, que se conocen como \textit{doble despacho}). Si está trabajando con dos jerarquías de tipos diferentes que están interactuando, entonces usted habrá de tener una llamada a un método polimórfico en cada jerarquía.  \newline

Aquí está un ejemplo de despacho múltiple: \newline

\begin{lstlisting} 
#: c11:PaperScissorsRock.py 
# Demonstration of multiple dispatching. 
from __future__ import generators 
import random 

# An enumeration type: 
class Outcome: 
  def __init__(self, value, name):  
    self.value = value 
    self.name = name 
  def __str__(self): return self.name  
  def __eq__(self, other): 
      return self.value == other.value 
      
Outcome.WIN = Outcome(0, "win") 
Outcome.LOSE = Outcome(1, "lose") 
Outcome.DRAW = Outcome(2, "draw") 

class Item(object): 
  def __str__(self):  
    return self.__class__.__name__  
    
class Paper(Item): 
  def compete(self, item): 
    # First dispatch: self was Paper 
    return item.evalPaper(self) 
  def evalPaper(self, item): 
    # Item was Paper, we're in Paper 
    return Outcome.DRAW 
  def evalScissors(self, item): 
    # Item was Scissors, we're in Paper 
    return Outcome.WIN 
  def evalRock(self, item): 
    # Item was Rock, we're in Paper 
    return Outcome.LOSE 
    
class Scissors(Item): 
  def compete(self, item):  
    # First dispatch: self was Scissors 
    return item.evalScissors(self) 
  def evalPaper(self, item): 
    # Item was Paper, we're in Scissors 
    return Outcome.LOSE 
  def evalScissors(self, item): 
    # Item was Scissors, we're in Scissors 
    return Outcome.DRAW 
  def evalRock(self, item): 
    # Item was Rock, we're in Scissors 
    return Outcome.WIN 
    
class Rock(Item): 
  def compete(self, item): 
    # First dispatch: self was Rock 
    return item.evalRock(self) 
  def evalPaper(self, item): 
    # Item was Paper, we're in Rock 
    return Outcome.WIN 
  def evalScissors(self, item): 
    # Item was Scissors, we're in Rock 
    return Outcome.LOSE 
  def evalRock(self, item): 
    # Item was Rock, we're in Rock 
    return Outcome.DRAW 
    
    def match(item1, item2): 
  print "\%s <--> \%s : \%s" \% ( 
    item1, item2, item1.compete(item2)) 
# Generate the items: 
def itemPairGen(n): 
  # Create a list of instances of all Items: 
  Items = Item.__subclasses__() 
  for i in range(n): 
    yield (random.choice(Items)(),  
           random.choice(Items)()) 
for item1, item2 in itemPairGen(20): 
  match(item1, item2) 
#:~ 
\end{lstlisting}

Esta fue una traducción bastante literal de la versión de Java, y una de las cosas que usted puede notar es que la información sobre las distintas combinaciones se codifica en cada tipo de \textbf{Item}. En realidad, termina siendo una especie de tabla excepto que se extiende a través de todas las clases. Esto no es muy fácil de mantener si alguna vez espera modificar el comportamiento o para añadir una nueva clase \textbf{Item}. En su lugar, puede ser más sensible a hacer la tabla explícita, así:  \newline

\begin{lstlisting} 
#: c11:PaperScissorsRock2.py 
# Multiple dispatching using a table 
from __future__ import generators 
import random 

class Outcome: 
  def __init__(self, value, name):  
    self.value = value 
    self.name = name 
  def __str__(self): return self.name  
  def __eq__(self, other): 
      return self.value == other.value 
      
Outcome.WIN = Outcome(0, "win") 
Outcome.LOSE = Outcome(1, "lose") 
Outcome.DRAW = Outcome(2, "draw") 

class Item(object): 
  def compete(self, item): 
    # Use a tuple for table lookup: 
        return outcome[self.__class__, item.__class__] 
  def __str__(self):  
    return self.__class__.__name__  
class Paper(Item): pass 
class Scissors(Item): pass 
class Rock(Item): pass 
outcome = { 
  (Paper, Rock): Outcome.WIN, 
  (Paper, Scissors): Outcome.LOSE, 
  (Paper, Paper): Outcome.DRAW, 
  (Scissors, Paper): Outcome.WIN, 
  (Scissors, Rock): Outcome.LOSE, 
  (Scissors, Scissors): Outcome.DRAW, 
  (Rock, Scissors): Outcome.WIN, 
  (Rock, Paper): Outcome.LOSE, 
  (Rock, Rock): Outcome.DRAW, 
} 
def match(item1, item2): 
  print "\%s <--> \%s : \%s" % ( 
    item1, item2, item1.compete(item2)) 
# Generate the items: 
def itemPairGen(n): 
  # Create a list of instances of all Items: 
  Items = Item.__subclasses__() 
  for i in range(n): 
    yield (random.choice(Items)(),  
           random.choice(Items)()) 
for item1, item2 in itemPairGen(20): 
  match(item1, item2) 
#:~ 
\end{lstlisting}

Es un tributo a la flexibilidad de los diccionarios que una tupla se puede utilizar como una clave tan fácilmente como un solo objeto. \newpage


\subsection*{Visitor, un tipo de despacho múltiple}
\label{subsec:vutddm}
\addcontentsline{toc}{subsection}{\nameref{subsec:vutddm}}


La suposición es que usted tiene una jerarquía primaria de clases que es fija; tal vez es de otro proveedor y no puede hacer cambios en esa jerarquía. Sin embargo, usted tenía como añadir nuevos métodos polimórficos a esa jerarquía, lo que significa que normalmente habría que añadir algo a la interfaz de la clase base. Así que el dilema es que usted necesita agregar métodos a la clase base, pero no se puede tocar la clase base. ¿Cómo se obtiene alrededor de esto?.     \newline

El patrón de diseño que resuelve este tipo de problemas se llama un "visitor : visitante" (el definitivo en el libro \textit{Design Patterns}), y se basa en el esquema de despacho doble mostrado en la última sección.    \newline

El patrón visitante le permite extender la interfaz del tipo primario mediante la creación de una jerarquía de clases por separado de tipo \textbf{Visitor} para virtualizar las operaciones realizadas en el tipo primario. Los objetos del tipo primario simplemente "aceptan" al visitante, a continuación, llamar a la función miembro de \textbf{visitor} dinámicamente enlazado.   \newline

\begin{lstlisting} 
#: c11:FlowerVisitors.py 
# Demonstration of "visitor" pattern. 
from __future__ import generators 
import random 

# The Flower hierarchy cannot be changed: 
class Flower(object):   
  def accept(self, visitor): 
    visitor.visit(self) 
  def pollinate(self, pollinator): 
    print self, "pollinated by", pollinator 
  def eat(self, eater): 
    print self, "eaten by", eater 
  def __str__(self):  
    return self.__class__.__name__
    
class Gladiolus(Flower): pass 
class Runuculus(Flower): pass 
class Chrysanthemum(Flower): pass  
class Visitor: 
  def __str__(self):  
    return self.__class__.__name__ 
class Bug(Visitor): pass 
class Pollinator(Bug): pass 
class Predator(Bug): pass 

# Add the ability to do "Bee" activities: 
class Bee(Pollinator): 
  def visit(self, flower): 
      flower.pollinate(self) 
      
# Add the ability to do "Fly" activities: 
class Fly(Pollinator): 
  def visit(self, flower): 
      flower.pollinate(self) 
      
# Add the ability to do "Worm" activities: 
class Worm(Predator): 
  def visit(self, flower): 
      flower.eat(self) 
      
def flowerGen(n): 
  flwrs = Flower.__subclasses__() 
  for i in range(n): 
    yield random.choice(flwrs)() 
    
# It's almost as if I had a method to Perform 
# various "Bug" operations on all Flowers: 
bee = Bee() 
fly = Fly() 
worm = Worm() 
for flower in flowerGen(10): 
  flower.accept(bee) 
  flower.accept(fly) 
  flower.accept(worm) 
#:~ 
\end{lstlisting}



\subsection*{Ejercicios}
\label{subsec:Ejercicios13}
\addcontentsline{toc}{subsection}{\nameref{subsec:Ejercicios13}}


1. Crear un entorno empresarial de modelado con tres tipos de \textbf{Inhabitant : Dwarf} (para Ingenieros), \textbf{Elf} (para los comerciantes) y \textbf{Troll} (para los administradores). Ahora cree una clase llamada \textbf{Project} que crea los diferentes habitantes y les lleva a \textbf{interact( )} entre sí utilizando despacho múltiple.   \newline

2. Modificar el ejemplo de arriba para hacer las interacciones más detalladas. Cada \textbf{Inhabitant} puede producir al azar un \textbf{Weapon} usando \textbf{getWeapon( )}: un \textbf{Dwarf} usa \textbf{Jargon} o \textbf{Play}, un \textbf{Elf} usa \textbf{InventFeature} o \textbf{SellImaginaryProduct}, y un \textbf{Troll} usa \textbf{Edict} y \textbf{Schedule}. Usted debe decidir qué armas "ganar" y "perder" en cada interacción (como en \textbf{PaperScissorsRock.py}). Agregar una función miembro \textbf{battle( )} a \textbf{Project} que lleva dos \textbf{Inhabitant}s y coinciden unos contra los otros. Ahora cree una función miembro \textbf{meeting( )} para \textbf{Project} que crea grupos de \textbf{Dwarf, Elf} y \textbf{Manager} y batallas contra los grupos entre sí hasta que sólo los miembros de un grupo se quedan de pie.  Estos son los "ganadores".    \newline

3. Modificar \textbf{PaperScissorsRock.py} para reemplazar el doble despacho con una búsqueda en la tabla. La forma más sencilla de hacerlo es crear un \textbf{Map} de \textbf{Map}s, con la clave de cada \textbf{Map} la clase de cada objeto. Entonces usted puede hacer la búsqueda diciendo: \par
\textbf{((Map)map.get(o1.getClass())).get(o2.getClass())}. \newline
Observe lo fácil que es volver a configurar el sistema. ¿Cuándo es más apropiado utilizar este enfoque vs. difíciles de codificación los despachos dinámicos? ¿Se puede crear un sistema que tiene la sencillez sintáctica de uso del despacho dinámico, pero utiliza una búsqueda en la tabla?   \newline

4. Modificar Ejercicio 2 para utilizar la técnica de tabla de consulta descrito en el Ejercicio 3.     \newline

\newpage
