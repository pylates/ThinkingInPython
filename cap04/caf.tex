\section*{3: Construyendo aplicaciones Framework}
\label{sec:caf}
\addcontentsline{toc}{section}{\nameref{sec:caf}}
% frameworks = MArco, entorno, estructura
% Preguntar! ¿aplicación Framework? ¿Framework de aplicación?
% Framework: conjunto de herramientas

Una aplicación framework le permite heredar de una clase o conjunto de clases y crear una nueva aplicación, reutilizando la mayor parte del código en las clases existentes y anular uno o más métodos con el fin de personalizar la aplicación a sus necesidades. Un concepto fundamental en el entorno de aplicación es el \textit{Template Method} (Método Plantilla) el cual normalmente se oculta debajo de las cubiertas e impulsa la aplicación  llamando a los diversos métodos en la clase base (algunos de los cuales usted ha reemplazado con el fin de crear la aplicación).    \newline

Por ejemplo, cuando se crea un applet, se está utilizando una aplicación framework: hereda de \textbf{JApplet} y luego reemplaza \textbf{init()}. El mecanismo applet (que es un método plantilla) se encarga del resto mediante la elaboración de la pantalla, el manejo del ciclo de eventos, los cambios de tamaño, etc.
        
        
\subsection*{\textit{Template Method} (Método Plantilla)}
\label{subsec:tm}
\addcontentsline{toc}{subsection}{\nameref{subsec:tm}}
        
        
        
Una característica importante del Método Plantilla es que está definido en la clase base y no puede ser cambiado. Este algunas veces es un método \textbf{Privado} pero siempre es prácticamente \textbf{final}. El llamado a otros métodos de la clase base (los que usted reemplaza) con el fin de hacer su trabajo, pero se le suele llamar sólo como parte de un proceso de inicialización (y por tanto el programador-cliente no necesariamente es capaz de llamarlo directamente).    \newline

 \begin{lstlisting}
#: c03:TemplateMethod.py 
# Simple demonstration of Template Method. 

class ApplicationFramework: 
  def __init__(self): 
    self.__templateMethod() 
  def __templateMethod(self): 
    for i in range(5): 
      self.customize1() 
      self.customize2() 
      
# Create a "application": 
class MyApp(ApplicationFramework): 
  def customize1(self): 
    print "Nudge, nudge, wink, wink! ", 
  def customize2(self):  
    print "Say no more, Say no more!" 
    
MyApp() 
#:~ 
\end{lstlisting}

El constructor de la clase base es responsable de realizar la inicialización necesaria y después de iniciar el "motor" (el método plantilla) que ejecuta la aplicación (en una aplicación GUI, este "motor" sería el
bucle principal del evento). El programador cliente simplemente proporciona definiciones para \textbf{customize1()}  y \textbf{customize2()} y la "aplicación" esta lista para funcionar.     \newline

Veremos \textit{Template Method} (Método plantilla) otras numerosas veces a lo largo del libro.

\subsection*{Ejercicios}
\label{subsec:exercises}
\addcontentsline{toc}{subsection}{\nameref{subsec:exercises}}

% .---- 16 deNoviembre
\begin{enumerate}[1.]
    \item Crear un framework que tome una lista de nombres de archivo en la línea de comandos. Este abre cada archivo, excepto el último para la lectura, y el último para la escritura. El framework procesará cada archivo de entrada utilizando una política indeterminada y escribir la salida al último archivo. Heredar para personalizar este framework para crear dos aplicaciones separadas:
    \begin{enumerate}[1)]
        \item Convierte todas las letras en cada archivo a mayúsculas.
        \item Busca los archivos de las palabras dadas en el primer archivo.
    \end{enumerate}
\end{enumerate}
