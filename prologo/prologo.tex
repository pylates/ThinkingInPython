\section*{Prólogo}
\label{sec:prolo}
\addcontentsline{toc}{section}{\nameref{sec:prolo}}

\textbf{El material de este libro inició en conjunción con el seminario de Java que yo he dado por varios años, un par de veces con Larry O’Brien, luego con Bill Venners. Bill y yo hemos dado muchas repeticiones de este seminario y a través de los años lo hemos cambiado ya que los dos hemos aprendido más acerca de patrones y sobre dar el seminario.}\newline

En el proceso, ambos hemos producido información más que suficiente para que cada uno de nosotros tengamos nuestros propios seminarios, un impulso que fuertemente hemos resistido porque juntos nos hemos divertido mucho dando el seminario. En numerosos lugares de Estados Unidos hemos dado el seminario, así como en Praga (donde nosotros intentamos hacer una miniconferencia en cada primavera, junto a otros seminarios). Ocasionalmente hemos dado  un seminario en Praga, pero esto es costoso y difícil de programar, ya que solo somos dos.\newline

Muchos agradecimientos a las personas que han participado en estos seminarios en los últimos años, y a Larry y a Bill, ya que me han ayudado a trabajar en estas ideas y a perfeccionarlas. Espero ser capaz de continuar para formar y desarrollar este tipo de ideas a través de este libro y del seminario durante muchos años por venir. \newline 

Este libro no parará aquí, tampoco. Originalmente, este material era parte de un libro de C++, luego de un libro de Java, entonces este fue separado de su propio libro basado en Java, y finalmente después de mucho examinar, decidí que la mejor manera para crear inicialmente mi escrito sobre patrones de diseño era escribir esto primero en Python (ya que sabemos que Python hace un lenguaje ideal de prototipos!) y luego traducir las partes pertinentes del libro de nuevo en la versión de Java.  He tenido la experiencia antes de probar la idea en un lenguaje más potente, luego traducir de nuevo en otro lenguaje, y he encontrado que esto es mucho más fácil para obtener información y tener la idea clara.\newline

Así que \textit{Pensando en Python} es, inicialmente, una traducción de \textit{Thinking in Patterns with Java}, en lugar de una introducción a Python (ya hay un montón de introducciones finas para este espléndido lenguaje). Me parece que este prospecto es mucho más emocionante que la idea de esforzarse a través de otro tutorial del lenguaje (mis disculpas a aquellos que estaban esperando para esto).

\newpage 